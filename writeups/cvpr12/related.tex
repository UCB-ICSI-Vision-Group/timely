\section{Related Work}
The literature on object detection is vast.
Here we briefly summarize detection work that sufficiently contextualizes our contribution.

An early success in efficient object detection used simple Haar features to build up a \emph{cascade} of classifiers, which then considered image regions in a sliding window regime~\cite{Viola2001}.
Although the simple features and classifiers of this method have been surpassed by more complex detectors, the idea of cascading classifier evaluations is still often used.

The best recent performance has come from detectors that use gradient-based features to represent objects as either a collection of local patches or as object-sized windows \cite{Dalal2005,Lowe2004}.
Usually, SVM classifiers are used to learn separating hyperplanes in such feature spaces between objects of a given class and all other possible contents of an image window.

Window proposal is often done exhaustively over the image space, in a ``sliding window'' fashion.
Using ``jump windows'' (window hypotheses voted on by local features) as region proposals is another common idea~\cite{Chum2007b,Vedaldi2009,Vijayanarasimhan2011}.
For local feature-based approaches, a bounded search over the space of all possible windows works reasonably well~\cite{Lampert2008b}---however, the method has not been able to obtain state-of-the-art performance.

Although none of the best-performing systems treat window proposal and evaluation as a closed-loop system, with feedback from evaluation to proposal, some work has been done in the area, mostly inspired by ideas from biological vision and attention research~\cite{Butko2009,Vogel2008,Paletta2005}.

Anytime performance in vision systems is a surprisingly little-explored idea.
A pioneering recent paper picks features with maximum value of information in a Hough-voting framework, and explicitly evaluates performance vs. time \cite{Vijayanarasimhan2010}.


A very interesting recent paper on Active Classification learned a dynamic policy for deploying classifiers in a multi-class, single-label task \cite{Gao2011}.
Their locally-weighted regression inference method has high runtime complexity.

Inherently multi-class detection has its own line of work, focusing largely on making detection time sublinear in the number of classes through sharing features~\cite{Torralba2007,Fan2005,Razavi2011}.
A recent post-processing extension to detection systems uses structured prediction to incorporate multi-class context as a principled replacement for the common post-processing step of non-maximum suppression~\cite{Desai2009}.

Context of course has a long history in vision.
One source of context is the scene or non-detector cues; for the PASCAL VOC, these are quantitatively considered in~\cite{Divvala2009}.
Another source is inter-object context, used for detection in a random field setting in~\cite{Torralba2004}.
A critical summary of the main approaches to using context for object detection is given in \cite{Galleguillos2010}.
