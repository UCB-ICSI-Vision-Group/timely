\documentclass{article} % For LaTeX2e
\usepackage{nips12submit_e,times}
\usepackage{graphicx}
\usepackage{amsmath,amssymb} % define this before the line numbering.
\usepackage{color}
\usepackage{subfig}
\usepackage{natbib}
\usepackage[pagebackref=true,breaklinks=true,colorlinks,bookmarks=false]{hyperref}

\newcommand{\argmax}{\operatornamewithlimits{argmax}}
\def\subsectionautorefname{section}
\definecolor{light-gray}{gray}{0.5}
\newcommand{\aside}[1]{\textcolor{light-gray}{\emph{#1}}}
\newcommand{\todo}[1]{\textcolor{red}{\emph{#1}}}
\newcommand{\cut}[1]{\textcolor{light-gray}{#1}}
\newcommand{\comment}[1]{}

\newcommand{\myparagraph}[1]{\vspace{-0.1cm}{\bf #1}}

\title{Timely Object Recognition}

\author{
Sergey Karayev\thanks{ Code available at http://sergeykarayev.com/work/timely/ } \\
UC Berkeley
\And
Tobias Baumgartner \\
Affiliation
\And
Mario Fritz \\
Address
\And
Trevor Darrell \\
UC Berkeley
}

\newcommand{\fix}{\marginpar{FIX}}
\newcommand{\new}{\marginpar{NEW}}

\nipsfinalcopy % Uncomment for camera-ready version

\begin{document}
\maketitle

\begin{abstract}
In a large visual multi-class detection framework, the timeliness of results can be crucial.
Our method for \emph{timely} multi-class detection aims to give the best possible performance at any single point after a start time; it is terminated at a deadline time.
Toward this goal, we formulate a dynamic, \emph{closed-loop} policy that infers the contents of the image in order to decide which detector to deploy next.
In contrast to previous work, our method significantly diverges from the predominant greedy strategies, and is able to learn to take actions with deferred values.
We evaluate our method with a novel \emph{timeliness} measure, computed as the area under an Average Precision vs. Time curve.
Experiments are conducted on the eminent PASCAL VOC object detection dataset.
If execution is stopped when only half the detectors have been run, our method obtains $66\%$ better AP than a random ordering, and $14\%$ better performance than an intelligent baseline.
On the timeliness measure, our method obtains at least $11\%$ better performance.
Our code, to be made available upon publication, is easily extensible as it treats detectors and classifiers as black boxes and learns from execution traces using reinforcement learning.
\end{abstract}

\section{Introduction}

In most real-world applications of object recognition, performance is time-sensitive and inherently tied to the many-class nature of the world.
In robotics, a small finite amount of processing power per unit time is all that is available for robust object detection if the robot is to usefully interact with humans.
In large-scale detection system deployments, such as for image search, results need to be obtained quickly per image as the number of images to process is large and growing.
In these cases, an acceptable answer at a reasonable time may be more valuable than the best answer given too late.

\begin{figure}[ht!]
\center{\includegraphics[width=0.86\linewidth]
    {../figures/figure1.pdf}}
  \caption{
A sample trace of our method.
At each time step beginning at $t=0$, potential actions are considered according to their \emph{value}, and the one with the highest value is picked.
The selected action is performed and returns observations, which influence the selection of the next action.
The final evaluation of a detection episode is the area of the AP vs. Time curve between the start and end times.
The value of an action is the expected value of the final evaluation if the action is taken and the policy continues to be followed, which allows actions without an immediate benefit to be scheduled.
}
  \label{fig:figure1}
\end{figure}

The conventional approach to evaluation of results on visual category
recognition does not consider efficiency and evaluates performance
indepednently across classes.  
We argue that the key to tackling such problems of dynamic recognition resource allocation is to start asking a new question:
\emph{What is the best performance we can get on a budget?}
We propose a new \emph{timeliness} measure of performance vs. time (shown in Figure~\ref{fig:figure1}), which evalutes the time efficiency of multi-class detection.

We present a method based on reinforcement learning that takes various detectors and classifiers as black boxes, and learns a dynamic policy for selecting detector and other actions to achieve the highest performance under this evaluation.
We are able to obtain better performance than baselines when there is less time available than is needed to run all detectors.

A hypothetical recognition system for a vision-based advertising deployment presents a motivational case study.
The system will have different accuracies for objects of different classes; detections will have different values based on confidence and class; and the queue of unprocessed images will vary in size.
The detection strategy to maximize profit in such an environment should depend on all of these variables.
We explore this scenario using the PASCAL VOC datasets and evaluation regimes, and show that our learned dynamic policy offers a better strategy than a random or optimal static ordering of detectors.

\section{Recognition Problems and Related Work}

We deal with a dataset of images $\mathcal{D}$, where each image $\mathcal{I}$ contains zero or more objects.
Each object is labeled with exactly one category label $k \in \{1, \dots, K\}$.

The multi-class, multi-label \textbf{classification} problem asks whether $\mathcal{I}$ contains at least one object of class $k$.
We write the ground truth for an image as $\mathbf{C}=\{C_1,\dots,C_K\}$, where $C_k \in \mathbb{B} = \{0,1\}$ is set to $1$ if an object of class $k$ is present.

The \textbf{detection} problem is to output a list of bounding boxes (sub-images defined by four coordinates), each with a real-valued confidence that it encloses a single instance of an object of class $k$, for each $k$.
The answer for a single class is given by an algorithm $\emph{detect}(\mathcal{I},k)$, which outputs a list of sub-image bounding boxes $B$ and their associated confidences.

The answer is evaluated by plotting precision vs. recall across dataset $\mathcal{D}$ (by progressively lowering the confidence threshold for a positive detection).
The area under the curve yields the Average Precision (AP) metric, which has become the standard evaluation for recognition performance on challenging datasets in vision \cite{pascal-voc-2010}.
A common measure of a correct detection is the PASCAL overlap: two bounding boxes are considered to match if they have the same label and the ratio of their intersection to their union is at least $\frac{1}{2}$.

To highlight the hierarchical structure of these problems, we note that (1) the confidences for each sub-image $b \in B$ may be given by $\emph{classify}(b,k)$; (2) correct answer to the detection problem also answers the classification problem.

Multi-class performance is evaluated by averaging the individual per-class AP values.
Since our use case is motivated by a potential advertising system, we generalize this metric to a weighted average, with the weights set by the \emph{values} of the classes.

\subsection{Related Work}
The literature on object recognition is vast.
Here we briefly summarize work that sufficiently contextualizes our contribution.

\paragraph{Single-class detection}
The best recent performance has come from detectors that use gradient-based features to represent objects as either a collection of local patches or as object-sized windows \cite{Dalal2005,Lowe2004}.
Classifiers are then used to distinguish between featurizations of a given class and all other possible contents of an image window.

Window proposal is often done exhaustively over the image space, as a ``sliding window''.
Using ``jump windows'' (window hypotheses voted on by local features) as region proposals is another common idea~\cite{Vedaldi2009,Vijayanarasimhan2011}.
For local feature-based approaches, a bounded search over the space of all possible windows works reasonably well~\cite{Lampert2008a}.

None of the best-performing systems treat window proposal and evaluation as a closed-loop system, with feedback from evaluation to proposal.
Some work has been done on this topic, mostly inspired by ideas from biological vision and attention research~\cite{Butko2009,Vogel2008,Paletta2005}.

\paragraph{Context}
Context has a foundational role in vision.
One source of context is the scene or other non-detector cues; the most common scene-level feature is the GIST \cite{Oliva2001a} of the image.
For the commonly used PASCAL VOC dataset \cite{pascal-voc-2010}, GIST and other sources of context are quantitatively explored in~\cite{Divvala2009}. 

Another source is inter-object context, shown to be useful for improving detection \cite{Torralba2004}.
A critical summary of the main approaches to using context for object and scene recognition is given in \cite{Galleguillos2010}.

\paragraph{Multi-Class Detection}
Work on inherently multi-class detection focuses largely on making detection time sublinear in the number of classes through sharing features \cite{Torralba2007,Fan2005,Razavi2011}.
A proposed post-processing extension to detection systems uses structured prediction to incorporate multi-class context as a principled replacement for the common post-processing step of non-maximum suppression \cite{Desai2009}.

\paragraph{Cascades}
An early success in efficient object detection used simple, fast features to build up a \emph{cascade} of classifiers, which then considered image regions in a sliding window regime \cite{Viola2001}.
Although the simple features and classifiers of this method have been surpassed by more complex detectors, the idea of cascading classifier evaluations is still often used.
Most recently, cyclic optimization has been applied to optimize cascades with respect to feature computation cost as well as classifier performance \cite{Chen2012}.

However, cascades are not dynamic policies---they cannot change the order of execution based on observations obtained during execution, which is our goal.

\paragraph{Anytime Algorithms and Active Classification}
Anytime performance in vision systems is a surprisingly little-explored idea.
A recent application to the problem of visual detection picks features with maximum value of information in a Hough-voting framework, and explicitly evaluates performance vs. time \cite{Vijayanarasimhan2010}. 

There has also been work on active classification \cite{Gao2011} and active sensing \cite{Yu2009}, in which intermediate results are considered in order to decide on the next classification step.
This line of work is closest to our approach.
Most commonly, the scheduling in these approaches is greedy with respect to some quantity such as expected information gain.

In contrast, we are able to learn policies that take actions without any immediate reward.

\section{Recognition Problems}

We deal with a dataset of images $\mathcal{D}$, where each image $\mathcal{I}$ contains at least one, and often multiple, objects.
Each object is labeled with exactly one category label $k \in \{1, \dots, K\}$.

The multi-class, multi-label \textbf{classification} problem asks whether $\mathcal{I}$ contains at least one object of class $k$.
The answer for a single label is given with a real-valued confidence by a function $\emph{classify}(\mathcal{I},k)$.
We write the ground truth for an image as $\mathbf{C}=\{C_1,\dots,C_K\}$, where $C_k \in \mathbb{B} = \{0,1\}$ is set to $1$ if an object of class $k$ is present.

The answer is evaluated by plotting precision vs. recall across dataset $\mathcal{D}$ (by progressively lowering the confidence threshold for a positive label) and integrating to yield the Average Precision (AP) metric, which has become the standard evaluation for recognition performance on challenging datasets \cite{pascal-voc-2010}.

\comment{
We can make the classification problem more difficult by posing the \emph{counting} problem, which asks how many objects of class $k$ are present in $\mathcal{I}$, for each $k$.
This setting is not commonly evaluated; we mention it for its usefulness in later exposition.
}

The \textbf{detection} problem is to output a list of bounding boxes (sub-images defined by four coordinates), each with a real-valued confidence that it encloses a single instance of an object of class $k$, for each $k$.
The answer for a single class is given by an algorithm $\emph{detect}(\mathcal{I},k)$, which outputs a list of sub-image bounding boxes $B$ and their associated confidences.

A common measure of a correct detection is the PASCAL overlap: two bounding boxes are considered to match if they have the same label and the ratio of their intersection to their union is at least $\frac{1}{2}$.
Again, Average Precision is the single-number metric for the performance of a detector.

As our task is fundamentally in \emph{multi-class} object detection, we rely on a slightly different evaluation than is commonly used (although it has precedent in \cite{Desai2009}).
Instead of pooling detections across images in the dataset, and considering classes individually, we pool detections across classes, but consider images individually, reporting results averaged across the dataset.

To highlight the hierarchical structure of these problems, we note that (1) the confidences for each sub-image $b \in B$ may be given by $\emph{classify}(b,k)$; (2) the correct answer to the detection problem also answers the classification problem. \comment{the counting and classification problems}

Our goal is a general recognition policy that outputs both classification and detection results; we evaluate on both tasks.

\section{Multi-class Recognition Policy}
\begin{figure}[h!]
\center{\includegraphics[width=0.66\linewidth]
    {figures/pomdp.pdf}}
  \caption{Summary of our approach to the problem. Our system has two major parts: (1) selecting an action by predicting its value; (2) updating the belief state with observations resulting from the action.}
  \label{fig:pomdp}
\end{figure}

Our goal is a multi-class recognition policy $\pi$ that takes an image $\mathcal{I}$ and outputs $\{\emph{classify}(1), \dots, \emph{classify}(K)\}$ and a list of multi-class detection results $\emph{detect}(I)$.

The policy repeatedly selects an action $a_i$ from a set of actions $\mathcal{A}$, executes it, potentially receives an observation $o_i$, and selects the next action.
The set of actions can include classifiers, detectors, or hybrid actions (detector followed by classification of its output).

A dynamic, or ``closed-loop,'' policy bases action selection on observations received from previous actions, exploiting the signal in inter-object and scene context for a maximally efficient path through the actions.
This is our goal, and what sets our formulation apart from multi-class systems that evaluate in a fixed order, such as simple cascades \cite{Viola2001} or \todo{add another one}.

Let $\mathcal{A}$ consist of $K$ detectors $L^\text{det}_i$.
We use one-vs-all deformable part-model classifiers on a HOG featurization of the image \cite{Felzenszwalb2010a}, with associated linear classification of the detections.

\comment{Let $\mathcal{A}$ consist of $K$ actions:
\begin{itemize}
  \item $K$ detectors $L^\text{det}_i$: one-vs-all deformable part-model classifiers on a HOG featurization of the image \cite{Felzenszwalb2010a}, with associated linear classification of the detections.
  \item $K$ classifiers $L^\text{gist}_i$: one-vs-all SVMs on the GIST feature \cite{Torralba2004}.
\end{itemize}}

\subsubsection{Time and Evaluation}
Each action $L$ has an expected cost $c(\cdot)$ of execution.
Depending on the setting, the cost can be defined in terms of algorithmic runtime analysis, an idealized property such as number of \emph{flops}, or simply the empirical runtime on specific hardware.
We take the empirical approach: every executed action advances $t$, the \emph{time into episode}, by its empirical runtime.

As shown in Figure~\ref{fig:evaluation}, the system is given two times: the setup time $T_s$ and deadline $T_d$.
From the setup time to the deadline, we want to obtain the best possible answer if stopped at any given time.
This corresponds to the general notion of Anytime algorithms, and is motivated by desired flexibility in the system.

A single-number metric that corresponds to this objective is simply the ratio of the area captured under the curve to the total area between the start and deadline bounds.
\comment{Just like the metric of Average Precision itself was motivated by the inadequacy of any single Precision-Recall operating point to describe the performance of a robust system, our proposed metric is motivated by the inadequacy of any single Performance vs. Time operating point.}
We evaluate policies by this more robust metric and not simply by the final performance at deadline time for the same reason that Average Precision is used instead of a fixed Precision vs. Recall point.

\subsubsection{Sequential Execution}
An action $a_i$ that consists of running a classifier $L_i$ returns a real-valued observation $o_i \sim P(O_i)$.
The state records the fact that $a$ has been taken by adding it to the initially empty set $\mathcal{O}$.
We refer to the current set of observations as $\mathbf{o} = \{o_i | L_i \in \mathcal{O}\}$.

We define the belief state $b$ of the decision process by the the distribution over class presence variables $P(\mathbf{C}) = P(C_1, \dots, C_K)$, where we write $P(C_k)$ to mean $P(C_k=1)$.
Additionally, $b$ records the time into episode $t$, and the set of executed actions $\mathcal{O}$ with corresponding observations $\mathbf{o}$.

A recognition \emph{episode} takes an image $\mathcal{I}$ and proceeds from the initial belief state $b^0$ and action $a^0$ to the next pair $b^1$, and so on until $t$ exceeds $T_d$.
At that point, the policy is terminated and a new episode begins on a new image.

The policy's performance at time $t$ is determined by the detection and classification observations that have been observed at the last belief state $b^j$ before that point.
For classification of unobserved classes, we treat the corresponding values of $P(\mathbf{C})$ as the set of confidence scores $\{\emph{classify}(k)\}$.
Detection results of unobserved classes are an empty set.

\todo{Is the above sufficiently clear?}

Our notation is summarized in \autoref{tab:notation}.

\begin{table}[h!]
\centering
\caption{Summary of the notation.}
\label{tab:notation}
\begin{tabular}{|l|l|}
  \hline
  $\mathcal{I}$ & image \\
  $C_k$         & presence of class $k \in \{1,\dots,K\}$ \\ 
  $t$           & time into episode \\ 
  $T_s$, $T_d$  & start and deadline times \\ 
  $b^j$         & belief state at step $j$ \\ 
  $\pi$         & policy function, $b \mapsto a \in \mathcal{A}$ \\
  $\mathcal{A}$ & set of actions $a_i \in \{\mathcal{L} \cup \mathcal{F}\}$\\ 
  $\mathcal{F}$ & set of featurization actions \\
  $\mathcal{L}$ & set of classification actions\\
  $o_i$         & a real-valued observation upon executing $a_i \in \mathcal{L}$\\
  $\mathcal{O}$ & set of executed actions\\
  $\mathbf{o}$  & set of observations $\{o_i | a_i \in \mathcal{O}\}$\\
  $c(a_i)$        & cost of executing $a_i$, in units of $t$\\
  \hline
\end{tabular}\end{table}

\section{Selecting actions} \label{sec:value}
\comment{Optimal performance results from becoming maximally certain of the correct values $C_i$ as quickly as possible (given the setup time).}
As our goal is to pick actions dynamically, we want to formulate a function $V(b,a)$  that assigns a value to a potential action, given the current state of the decision process.
We can then define the policy as simply the untaken action with the maximum value:
\begin{align}
\pi(b) = \argmax_{a_i \in \mathcal{A} \setminus \mathcal{O}} V(b,a_i)
\end{align}

For this policy to be \emph{closed-loop}, the observations $o_i$ generated by taking an action $a_i$ need to update $b$ and thus influence the selection of the next action.
Figure~\ref{fig:pomdp} visualizes such an execution process.

Before we discuss how to set $V(b,a_i)$ such that our policy obtains best performance under our evaluation, we present our model for updating the belief state with observations.

\subsection{The model of the belief state}
\begin{figure}[h!]
\centering
\includegraphics[width=0.56\linewidth]{figures/inf_model_mrf.pdf}
\caption{
We present a point in the middle of the decision process for 3 classes.
Some classifiers have already been observed.
}
\label{fig:model}
\end{figure}

The quantities that may be useful to us for selecting which actions to deploy are the probabilities and entropies of the class presence variables $C_k$.
These allow us to look for the most probable classes given the observations.
When the policy starts, the model should present the prior distributions $P(C_k)$; as observations are accrued, the model should present the updated conditionals $P(C_k|\mathbf{o})$.

We employ a fully-connected Markov Random Field (MRF), as shown in Figure~\ref{fig:model}.
The parameters are trained on fully-observed data.
Exact inference is of course intractable in a general model of this type, and quite slow for a model of our size.
Instead, we use Loopy Belief Propagation, which provides no general convergence guarantees but has been shown to work well empirically \todo{cite}.
The implementations details are given further in \autoref{sec:implementation}.

\subsection{Policy parametrization and more reward functions}

At each time step, our system is evaluated by properties of the state $b$ (the list of detections and the classification outputs).
The final evaluation metric is a function of the history of execution $h^0=b^0,b^1,\dots,b^J$, with $J$ being the last step of the process with $t \le T_d$.

Ideally, the value function for a point in the decision process $b^j$ should give the expected value of the final evaluation metric, over all possible histories starting at point $j$:
\begin{align}
V(b^j,a_i) = \mathbb{E}_{h^j \sim P(h^j|b,a_i)}[R(h^j)]
\end{align}

The reward function $R(h^j)$ assigns a real-valued score to a history.
We have considerable flexibility in defining $R$; it does not necessarily have to be directly tied to the final evaluation.

\todo{here introduce the view of learning parameters given the observed rewards}

Let's say that given deadline $T_d$ and some image, the policy had time to take $J$ actions.
The total expected reward of a policy $\pi$ is then defined as the sum of rewards
\begin{equation}
\mathbb{E}_\pi[\sum_{j=0}^J R(b_j)]
\end{equation}

The reward function does not necessarily have to be related to the evaluation of the system.

\subsubsection{Slope of performance vs. time}
\begin{figure}[htb]
  \centering
  \includegraphics[width=0.56\linewidth]{figures/apvst_expl.pdf}
  \caption{A per-action greedy value function that corresponds to the maximization of our objective function is the area of the horizontal slice under the curve due to the action. The figure shows this analysis for the action highlighted in orange.}
  \label{fig:rewards}
\end{figure}

Remembering that the final evaluation of a policy consists of the area under the performance vs. time curve, we formulate per-action rewards such that their addition results in precisely this quantity.
Specifically, as shown in Figure~\ref{fig:rewards}, we define the reward of an action as
\begin{equation}\label{eq:advanced}
\Delta AP (t_T-\Delta t)
\end{equation}
where $t_T$ is the time left until deadline, and $\Delta t$ and $\Delta AP$ are the time taken and AP change produced by the action.

The equation breaks down into a term to maximize, $\Delta AP t_T$, and a term to minimize, $\Delta AP \Delta t$.
This agrees with the intuition that to capture the most area under the curve, the slope needs to be maximized at each point.
Additionally, the equation shows that if $\Delta t$ exceeds $t_T$, the value of taking the action is negative.

At each step, we want to pick the action that maximizes the expected rewards until the end of the episode.
We therefore define our policy as
\begin{equation}
\pi(b) = \argmax_{a \in \mathcal{A} \setminus \mathcal{O}} \theta^\top \phi(b,a)
\end{equation}

Learning this accurately is the domain of Reinforcement Learning research, and is generally intractable for large-scale POMDPs.

First, we seek to maximize the expected next-stage reward, and manually construct a value function such that picking the action with maximal value achieves this.
Next, we note that the manually constructed value function can be represented as a scalar product of a parameter vector with a featurization of the belief state; we learn the weights by regression to the actual next-stage reward from running many episodes.
Lastly, we use a reinforcement learning technique to learn the weights such that the expected rewards over the whole episode are maximized.

\subsubsection{Learning the greedy policy}
We note that we can represent the one-step greedy policy value function as a scalar product 

\subsubsection{Learning the proper policy}
\note{I'm considering two options here: (1) log-linear representation and policy gradient algorithm; (2) least-squares policy iteration.}

\paragraph{Policy gradient algorithm}
We redefine our policy as picking the mode of a log-linear parametrized distribution:
\begin{equation}
P(a|b;\theta) = \frac{\exp(\theta^T \phi(b,a))}{\sum_{a' \in \mathcal{A}} \exp(\theta^T \phi(b,a')}
\end{equation}

Basically stochastic gradient ascent on the estimate of the reward accrued by an episode.

This approach has been shown to scale to large problems in \cite{Branavan2009}, although it has been proven to converge only to local maxima and only for MDPs \cite{Sutton2000}.

\note{Does the constraint that no action is taken more than once affect this algorithm?}

\paragraph{Least-squares Policy Iteration}
An efficient form of temporal-difference learning used in \cite{Kwok2004}.
Policy remains parameterized as it was.

\section{Implementation Details and Evaluation} \label{sec:implementation}
We use FastInf \cite{Jaimovich2010} for learning and inference in our MRF model.
The real-valued classifier responses $o_i$ are discretized into data-dependent bins for efficiency.

\section{Evaluation} \label{sec:evaluation}
- show that the greedy policy works better than random and better than fixed-order

- show that the reinforcement learning policy works better than greedy, at least for the non-infogain rewards.

\subsection{Classification}

\subsection{Detection}
We also evaluate using the task of detection (see \autoref{sec:problem}).

\section{Conclusion}
We presented a method for learning ``closed-loop'' policies for multi-class object recognition, given existing object detectors and classifiers and a metric to optimize.
The method learns the optimal policy using reinforcement learning, by observing execution traces in training.
If detection on an image is cut off after only half the detectors have been run, our method does $66\%$ better than random, and $14\%$ better than an intelligent baseline. In particular, our method learns to take action with no intermediate reward in order to improve the overall performance of the system.

As always with reinforcement learning problems, defining the reward function requires some manual work.
Here, we derive it for the novel detection AP vs. Time evaluation that we suggest is most useful for evaluating efficiency in recognition.
Although computation devoted to scheduling actions is much less significant than the computation due to running the actions, an interesting future research direction is to explicitly consider this decision-making cost.
At time of publication, we will release the code of our modular framework, that allows for easy incorporation of other classifiers and detectors on different tasks and rewards.

% \subsubsection*{Acknowledgments}
% Use unnumbered third level headings for the acknowledgments. All
% acknowledgments go at the end of the paper. Do not include 
% acknowledgments in the anonymized submission, only in the 
% final paper. 

\renewcommand\bibsection{\subsubsection*{\refname}}
\bibliographystyle{unsrt}
\small{
  \bibliography{../sergeyk_Timely}
}

\end{document}
