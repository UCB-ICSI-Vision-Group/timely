\section{Related Work}
The literature on object detection is vast.
Here we briefly summarize work relevant to our contribution.

An early success in efficient object detection used simple Haar features to build up a \emph{cascade} of classifiers, which then considered image regions in a sliding window regime~\cite{Viola2001}.
Plentiful later work improved the behavior of the cascade while maintaining the basic idea~\cite{Bourdev2005}.
This detection method is fast, but the simple features and classifiers used have not led to the best performing detectors.

The best performance has recently come from detectors that use gradient-based features to represent either local patches or object-sized windows.
If local patches are used, it appears important to include additional feature channels such as shape and color.
If object-sized windows are used, finer-scale ``parts'' in object representation boost performance significantly.

The region proposal is usually done exhaustively over the image space, and doing so in an efficient order has not received much attention in the literature.
Using ``jump windows'' (window hypotheses voted on by local features) as region proposals is one common idea~\cite{Chum2007b,Vijayanarasimhan2011}.
For local features, a bounded search over the space of all possible windows works well (especially for single-object detection)~\cite{Lampert2008b}.
The method requires derivation of bounds, which have not yet been developed for the best-performing HOG-based detectors \cite{Dalal2005, Felzenszwalb2010a}, limiting the method's usefulness in state-of-the-art systems.
%Similar ideas have been formulated in a decision theoretic way~\cite{Sznitman2010}.

One recent idea is to use a class-independent measure of ``objectness'' to reject much of the hypothesis space before running any class-specific detectors~\cite{Alexe2010,Endres2010}.
However, these approaches have not been shown to be actually faster than exhaustive evaluation, due to an expensive proposal step~\cite{Endres2010}.

Evaluation of regions involves feature computation, which is a time consuming step.
Efficient feature computation for HOG-based detectors has been explored in~\cite{Dollar2010}.
Another idea is coarse-to-fine model evaluation.
A recent work applies coarse-to-fine evaluation and adds a feedback arrow from post-processing to proposals, obtaining an order of magnitude speedup in the deformable part models framework while losing some accuracy~\cite{Pedersoli2011}.

Other systems that add feedback arrows to the conceptual diagram of Proposals -> Classifiers -> Post-processing are often inspired by biological vision and sequential decision process ideas~\cite{Butko2009,Vogel2008,Paletta2005}.
These works are very close to our motivation.
Another piece of motivation is Anytime performance, which is a largely unexplored idea for vision systems.
A pioneering recent paper picks features with maximum value of information in a Hough-voting framework, and explicitly evaluates itself with regard to time \cite{Vijayanarasimhan2010}.

Multi-class detection has its own line of work, focusing largely on detection time sublinear in the number of classes through sharing features~\cite{Torralba2007,Fan2005,Razavi2011}.
An interesting reinforcement learning approach in a cascade framework is taken in~\cite{Isukapalli2006}.

Context has a long history in vision.
One source of context is the scene or non-detector cues; for the PASCAL VOC, these are quantitatively considered in~\cite{Divvala2009}.
Another source is inter-object context, used for detection in a random field setting in~\cite{Torralba2004}.

Attention modeling is a long-standing research problem in psychophysics, resulting in a few key concepts: saliency, bottom-up vs. top-down effects, and using saccades as a proxy~\cite{Itti2001a,Chikkerur2010,Judd2009}.
The concept of saliency has been used for object classification, and has been explored to no great effect for detection~\cite{Kanan2010}.

The two papers closest to our contribution are principled multi-class structured prediction~\cite{Desai2009} and the detection under bounded resources work~\cite{Vijayanarasimhan2010}.
We build atop the first work, which assumes that all classes have been evaluated at all regions.
We share the motivation of Anytime performance with the second work, but in a significantly more powerful detection regime.
Although we rely on the commonly used deformable parts model detector, our method can use any feature extraction and classification method.
